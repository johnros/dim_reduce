\renewcommand*{\marginfont}{\scriptsize }

\newcommand{\reals}{\mathbb{R}} % the set of real numbers
\newcommand{\argmin}[2]{\textstyle{\mathop{argmin}_{#1}}\set{#2}} % The argmin operator
\newcommand{\argmax}[2]{\textstyle{\mathop{argmax}_{#1}}\set{#2}} % The argmin operator
\newcommand{\manifold}{\mathcal{M}} % A manifold.
\newcommand{\project}{\hookrightarrow} % The orthogonal projection operator.
\newcommand{\projectMat}{H} % A projection matrix.
\newcommand{\encode}{E} % a linear encoding matrix
\newcommand{\decode}{D} % a linear decoding matrix
\DeclareMathOperator{\Tr}{Tr}
\newcommand{\set}[1]{\{ #1 \}} % A set
\newcommand{\setII}[1]{\left\{ #1 \right\}} % A set
\newcommand{\rv}[1]{\mathbf{#1}} % A random variable
\newcommand{\x}{\rv x} % The random variable x 
\newcommand{\y}{\rv y} % The random variable x 
\newcommand{\U}{\rv u} % The random variable x 
\newcommand{\T}{\rv t} % The random variable x 
\newcommand{\X}{\rv X} % The random variable x 
\newcommand{\Y}{\rv Y} % The random variable y
\newcommand{\expect}[1]{\mathbf{E}\left[ #1 \right]} % The expectation operator
\newcommand{\expectg}[2]{\mathbf{E}_{\rv{#1}}\left[ \rv{#2} \right]} % An expectation w.r.t. a particular random variable.
\newcommand{\expectn}[1]{\mathbb{E}\left[#1\right]} % The empirical expectation
\newcommand{\cov}[1]{\mathbf{Cov} \left[ #1 \right]} % The expectation operator
\newcommand{\var}[1]{\mathop{Var} \left[ #1 \right]} % The expectation operator
\newcommand{\covn}[1]{\mathbb{Cov} \left[ #1 \right]} % The expectation operator
\newcommand{\gauss}[1]{\mathcal{N}\left(#1\right)} % The gaussian distribution
\newcommand{\cdf}[2]{F_{#1} (#2)} % The CDF function
\newcommand{\survive}[2]{S_{#1} (#2)} % The survival function
\newcommand{\hazard}[2]{h_{#1} (#2)} % The survival function
\newcommand{\cuhazard}[2]{H_{#1} (#2)} % The survival function
\newcommand{\cdfn}[2]{\mathbb{F}_{#1}(#2)} % The empirical CDF function
\newcommand{\icdf}[2]{F_\rv{#1}^{-1} (#2)} % The invecrse CDF function
\newcommand{\icdfn}[2]{\mathbb{F}^{-1}_{#1}(#2)} % The inverse empirical CDF function
\newcommand{\pdf}[2]{p_{#1} (#2)} % The CDF function
\newcommand{\prob}[1]{P\left( #1 \right)} % the probability of an event
\newcommand{\dist}{P} % The proabaiblity distribution
\newcommand{\density}{p}
\newcommand{\entropy}{H} % entropy
\newcommand{\mutual}[2]{I\left(#1;#2\right)} % mutual information
\newcommand{\norm}[1]{\Vert #1 \Vert} % The norm operator
\newcommand{\normII}[1]{\norm{#1}_2} % The norm operator
\newcommand{\normI}[1]{\norm{#1}_1} % The norm operator
\newcommand{\normF}[1]{\norm{#1}_{Frob}} % The Frobenius matrix norm
\newcommand{\ones}{\textbf{1}} % Vector of ones.
\newcommand{\lik}{\mathcal{L}} % The likelihood function
\newcommand{\loglik}{L} % The log likelihood function
\newcommand{\loss}{l} % A loss function
\newcommand{\lossII}{\prescript{}{2}{l}} % A loss function
\newcommand{\risk}{R} % The risk function
\newcommand{\riskn}{\mathbb{R}} % The empirical risk
\newcommand{\riskII}{\prescript{}{2}{R}} % The empirical risk
\newcommand{\risknII}{\prescript{}{2}{\mathbb{R}} } % The empirical risk
\newcommand{\noisen}{\mathbb{G}} % The empirical noise process
\newcommand{\deriv}[2]{\frac{\partial #1}{\partial #2}} % A derivative
\newcommand{\hyp}{f} % A hypothesis
\newcommand{\hypclass}{\mathcal{F}} % A hypothesis class
\newcommand{\hilbert}{\mathcal{H}}
\newcommand{\rkhs}{\hilbert_\kernel} % A hypothesis class
\newcommand{\normrkhs}[1]{\norm{#1}_{\rkhs}} % the RKHS function norm
\newcommand{\rank}{q} % A subspace rank.
\newcommand{\dimy}{K} % The dimension of the output.
\newcommand{\latent}{\rv{s}} % latent variables matrix
\newcommand{\latentn}{S} % latent variables matrix
\newcommand{\loadings}{A} % factor loadings matrix
\newcommand{\rotation}{R}  % rotation matrix
\newcommand{\similaritys}{\mathfrak{S}} % a similarity graph
\newcommand{\similarity}{s} % A similarity measure.
\newcommand{\dissimilarity}{d} % A dissimilarity measure.
\newcommand{\dissimilaritys}{\mathfrak{D}} % a dissimilarity graph
\newcommand{\scalar}[2]{\left< #1,#2 \right>} % a scalar product
\newcommand{\aka}{{a.k.a.\ }}
\newcommand{\Aka}{{A.k.a.\ }}









\theoremstyle{plain}
\newtheorem{theorem}{Theorem}[section]
\newtheorem*{theorem*}{Theorem}
\newtheorem{lemma}{Lemma}[section]
\newtheorem*{lemma*}{Lemma}
\newtheorem{prop}{Proposition}[section]
\newtheorem{cor}{Corollary}[section]


\theoremstyle{definition}
\newtheorem{definition}{Definition}
\newtheorem{remark}{Remark}
%\newtheorem{think}{Gedankenexperiment}
%\newtheorem*{think}{Think about it \faLightbulbO}

\newtheorem{example}{Example}

\newenvironment{think}
	{
		\bigskip
		\begin{tcolorbox}
		\paragraph{Think about it.}
	}{
		\end{tcolorbox}
}

\newenvironment{extra}
{
	\bigskip
	\begin{tcolorbox}
		\paragraph{Extra Information.}
	}{
	\end{tcolorbox}
}



% Custom commands

\newcommand{\naive}{na\"{\i}ve }
\newcommand{\Naive}{Na\"{\i}ve }
\newcommand{\andor}{and\textbackslash or }
\newcommand{\erdos}{Erd\H{o}s }
\newcommand{\renyi}{R\`enyi }
